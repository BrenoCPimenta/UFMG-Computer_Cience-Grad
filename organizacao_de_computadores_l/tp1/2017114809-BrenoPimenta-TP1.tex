%%% Template originaly created by Karol Kozioł and ShareLaTeX and modified by Breno Pimenta

\documentclass[a4paper,11pt]{article}

\usepackage[T1]{fontenc}
\usepackage[utf8]{inputenc}
\usepackage{graphicx}
\usepackage{xcolor}

\renewcommand\familydefault{\sfdefault}
\usepackage{tgheros}

\usepackage{amsmath,amssymb,amsthm,textcomp}
\usepackage{enumerate}
\usepackage{multicol}
\usepackage{tikz}

\usepackage{geometry}
\geometry{total={210mm,297mm},
left=25mm,right=25mm,%
bindingoffset=0mm, top=20mm,bottom=20mm}


\linespread{1.3}

\newcommand{\linia}{\rule{\linewidth}{0.5pt}}

% custom theorems if needed
\newtheoremstyle{mytheor}
    {1ex}{1ex}{\normalfont}{0pt}{\scshape}{.}{1ex}
    {{\thmname{#1 }}{\thmnumber{#2}}{\thmnote{ (#3)}}}

\theoremstyle{mytheor}
\newtheorem{defi}{Definition}

% Titulo customizado
\makeatletter
\renewcommand{\maketitle}{
\begin{center}
\vspace{2ex}
{\huge \textsc{\@title}}
\vspace{1ex}
\\
\linia\\
\@author \hfill \@date
\vspace{4ex}
\end{center}
}
\makeatother
%%%

% footers and headers customizados:
\usepackage{fancyhdr}
\pagestyle{fancy}
\lhead{}
\chead{}
\rhead{}
%\lfoot{Assignment \textnumero{} 1}
\cfoot{}
\rfoot{Page \thepage}
\renewcommand{\headrulewidth}{0pt}
\renewcommand{\footrulewidth}{0pt}
%

% code listing settings
\usepackage{graphicx,color}
\usepackage{xcolor}
\usepackage{booktabs}
\usepackage{listings}


\lstdefinelanguage{riskV}{
  sensitive = true,
  keywords=[1]{text, data, string},
  keywords = [2]{la, lw, and, andi, add, addi, ecall, jal,
  jalr, beq, bne, blt, bge, mul, sw},
  keywords = [3]{x0, x1, x2, x3, x4, x5, x6, x7, x8, x9,
   x10,x11, x12, x13, x14, x15, x16, x17, x18, x19, x20,
   x21, x22, x23, x24, x25, x26, x27, x28, x29, x30, x31, pc},
  keywordstyle=[1]\color{green},
  keywordstyle=[2]\color{blue},
  keywordstyle=[3]\bfseries,
  numbers=left,
  numberstyle=\scriptsize,
  stepnumber=1,
  numbersep=8pt,
  showstringspaces=false,
  breaklines=true,
  frame=top,
  comment=[l]{\#},
  %comment=[l]{//},
  %morecomment=[s]{/*}{*/},
  commentstyle=\color{purple}\ttfamily,
  stringstyle=\color{red}\ttfamily,
  morestring=[b]',
  morestring=[b]"
}

%%%----------%%%----------%%%----------%%%----------%%%

\begin{document}

\title{Assignment \textnumero{} 1}

\author{Breno de Castro Pimenta, RA: 2017114809}

\date{06/09/2019}

\maketitle

\section*{Problem 1: Evens and Odds}

Programa que confere se o número é par ou ímpar.\\
Caso seja Par retorna 0.\\
Caso seja Ímpar retorna 1.\\

\begin{figure}[htb]
\begin{small}
\begin{lstlisting}[language=riskV]
.data
  input:.word 2227
 
.text
main:
  la x6, input		#Armazenando endereco do input
  lw x5, 0(x6)		#Armazenando valor do input
  andi x10, x5, 1	#Buscando o bit menos significativo
  addi a1, x10, 0	#Armazenando resposta em a1
  addi a0, x0, 1 	#Transformando ecal na saida
  ecall			#Imprimindo resultado
  jal  zero, end	#Voltando para a ordem de execucao

end:
\end{lstlisting}
\end{small}
\end{figure} 

\pagebreak
\section*{Problem 2: Factorial}

O programa retorna o fatorial de um dado número.

\begin{figure}[htb]
\begin{small}
\begin{lstlisting}[language=riskV]
.data
  input:.word 4
 
.text
main:
  addi sp, sp, -16	 #Alocando espaco na pilha.
  sw x5, 0(sp)     	 #Salvando valores de 
  sw x6, 4(sp)		 #registradores que serao utilizados.
  sw x7, 8(sp)
  sw x1, 12(sp)
  
  la x6, input		 #Armazenando endereco do input.
  lw x5, 0(x6)		 #Armazenando valor do input.
  add x10, x5, x0	 #Armazenando em x10 o input.
  addi x7, x0, 1   	 #Colocando x7 com valor de 1.
  
  jal x1, FACT		 #Inicia-se o fatorial.
  
  addi a1, x10, 0	 #Armazenando resposta em a1.
  addi a0, x0, 1 	 #Transformando ecal na saida.
  ecall				 #Imprimindo resultado.
  
  lw x5, 0(sp)   	 #Restaurando os valores  dos
  lw x6, 4(sp)		 #registradores utilizados.
  lw x7, 8(sp)
  lw x1, 12(sp)
  addi sp,sp, 16  	 #Desalocando os espacos na pilha.
  jal  zero, end	 #Voltando para a ordem de execucao.
  
FACT:
  mul x10, x10, x7 	 #Multiplicamos x7 pelo resultado.
  addi x7, x7, 1	 #Aumentamos o x7.
  blt x7, x5, FACT 	 #Se x7>=x5 paramos a conta.
  jalr x0, 0(x1)	 #Volta para a main.
  
end:
\end{lstlisting}
\end{small}
\end{figure} 


\pagebreak
\section*{Problem 3: Permutation}
O programa verifica se o um vetor é a permutacao de outro.\\
Caso seja retorna 1, caso contrario 0.


\begin{small}
\begin{lstlisting}[language=riskV]
.data
  inputVetorOrdenado: .word 1,2,3,4 	# Usuario input vetor ordenado
  inputVetorDesordenado:.word 4,2,3,1 	# Usuario input vetor desordenado
  inputTamanhoVetor: .word 4 		# Usuario input tamanho vetor
    
.text
main: 
  addi sp, sp, -32     #Alocando espaco na pilha
  sw x5, 0(sp)         #Salvando valor de registrador que sera usado
  sw x6, 4(sp)
  sw x7, 8(sp)
  sw x18, 12(sp)
  sw x28, 16(sp)
  sw x29, 20(sp)
  sw x30, 24(sp)
  sw x31, 28(sp)
  
  la x5, inputTamanhoVetor 	#Armazenando endereco do inputTamanhoVetor
  lw x5, 0(x5)			#Armazenando valor do inputTamanhoVetor
  la x6, inputVetorOrdenado	#Armazenando endereco do inputVetorOrd.
  la x7, inputVetorDesordenado  #Armazenando endereco do inputVetorDesor.
  
  addi x28, x0, -1 		#Reinicializando x28 como -1.
  addi x10, x0, 1 		#Inicializando x10 como 1.
  addi x6, x6, -4 		#Vetor ordenado -1 posicao
  beq x0, x0, comparando	#Iniciando a comparacao

   
comparando:
  addi x28, x28, 1		#Iterando de 1 em 1
  beq x28, x5, final		#VerificandoSeTerminou
  addi x6, x6, 4		#Proxima posicao no vetor
  lw x29, 0(x6)			#Valor do vetor ordenado
  addi x30, x7, -4		#Reinicializando x30 uma posicao antes  
  add x18, x0, x0		#Reinicializando x18 em 0.
  beq x0, x0, comparacaoInterna
    
comparacaoInterna:
  addi x30, x30, 4 		#Somando um na posicao do vetor Desordenado
  lw x31, 0(x30)		#Valor do vetor desordenado
  beq x31, x29 comparando 	#Compara se sao iguais
  addi x18, x18, 1 		#Iterando 1
  beq x18, x5, naoOrdenado	#Comparou com todo o vetor e terminou
  bne x18, x5, comparacaoInterna 	#Continua comparando
    
naoOrdenado:
  add x10, x0, x0	#Atribuindo valor zero por nao estar ordenado
  beq x0, x0, final	#Pulando para o final
 	
final:
  addi a1, x10, 0	#Armazenando resposta em a1
  addi a0, x0, 1 	#Transformando ecal na saida
  ecall			#Imprimindo resultado
  
  lw x5, 0(sp)    	#Restaurando valor de registrador utilizado
  lw x6, 4(sp)
  lw x7, 8(sp)
  lw x18, 12(sp)
  lw x28, 16(sp)
  lw x29, 20(sp)
  lw x30, 24(sp)
  lw x31, 28(sp)
  addi sp,sp, 32     	#Desalocando espaco na pilha
  jal  zero, end	#Voltando para a ordem de execucao
  
end:
\end{lstlisting}
\end{small}

\end{document}
